The relationship between geometry and computational hardness has been recognized in the field of lattice-based cryptography. In a lecture delivered at the Cryptography Boot Camp organized by the Simons Institute in 2015~\cite{vaikuntanathan2015}, Vinod Vaikuntanathan discussed the Short Integer Solutions (SIS) problem, a cornerstone of modern lattice-based cryptography. Initially presenting an underdetermined system of linear equations, Vaikuntanathan noted that finding solutions is straightforward using classical linear algebra techniques such as Gaussian elimination. However, he highlighted that introducing a geometric constraint---specifically, the requirement that solutions must be ``short''---transforms the problem into one that is ``insanely hard.''

This distinction points to a deeper insight: while not all geometric problems in lattices are computationally hard, the hardness of all difficult lattice problems fundamentally stems from their geometric nature. This observation emphasizes geometry as a source of computational complexity in lattice-based problems, a foundation upon which many cryptographic systems are built today.

Motivated by this insight, and during a discussion at the International Conference on Geometric Algebra held in Denver in 2022, we posed the following question to Leo Dorst~\cite{dorstprofile}: \textit{``If the source of hardness in lattice problems is geometric, should we not use a geometric language to address them?''} Dorst's informal but compelling reply was: \textit{``If GA is not the language for that, nothing else is.''}

This intersection of ideas (the geometric origin of hardness in cryptographic problems and the expressive power of Geometric Algebra) motivates the present investigation. Although our long-term interest surely includes understanding the potential impact of GA on the security aspects of cryptographic hardness assumptions, the present work focuses on investigating performance, compactness, and expressiveness. These properties are of particular relevance to computationally intensive areas of cryptography, such as homomorphic encryption, and to resource-demanding domains in machine learning, notably large-scale model training and high-dimensional data transformations. 

In this work, we explore whether Geometric Algebra not only offers richer expressiveness but also delivers concrete computational advantages when applied to operations fundamental to cryptography and artificial intelligence.