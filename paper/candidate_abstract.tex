A considerable amount of cryptography and artificial intelligence (AI) systems fundamentally resort to linear algebra for core computations. Although classical vector and matrix methods are well-established, Geometric Algebra (GA) offers a richer mathematical framework capable of modeling geometric structures, transformations, and interactions more naturally. Despite GA’s theoretical appeal such as compactness, composability, and geometric fidelity, concrete demonstrations of its computational benefits in the aforementioned fields have not yet reached mainstream status. In this paper, we present reproducible benchmarks and analyses showing that GA can deliver both performance improvements and enhanced expressiveness compared to classical approaches. We focus on operations critical to cryptographic and AI applications, including multivector products, rotations, and batched transformations. Our results directly contribute to the idea that many hard problems in cryptography possess an underlying geometric structure and that GA, being the most natural formalism for geometric computation, offers tangible computational advantages. Furthermore, even in machine learning models where the geometric nature of problems is less explicit, GA’s performance, compactness, and expressive capabilities offer valuable computational benefits. We conclude by discussing pathways for broader integration of GA into cryptographic primitives, machine learning models, and even general-purpose computations in high-performance workflows.